\documentclass[thesis=B,czech]{FITthesis}[2012/06/26]

\usepackage[utf8]{inputenc}

\usepackage{graphicx} %graphics files inclusion
\usepackage{amsmath} %advanced maths
\usepackage{amsthm}
\usepackage{dirtree} %directory tree visualisation
\usepackage{algorithm}
\usepackage{algpseudocode}
\usepackage{amssymb} %additional math symbols
\usepackage{bm}

% % list of acronyms
% \usepackage[acronym,nonumberlist,toc,numberedsection=autolabel]{glossaries}
% \iflanguage{czech}{\renewcommand*{\acronymname}{Seznam pou{\v z}it{\' y}ch zkratek}}{}
% \makeglossaries

\newcommand{\tg}{\mathop{\mathrm{tg}}} %cesky tangens
\newcommand{\cotg}{\mathop{\mathrm{cotg}}} %cesky cotangens
\newcommand{\Break}{\State \textbf{break} } %prikaz break v algoritmech

\floatname{algorithm}{Algoritmus}

\theoremstyle{definition}
\newtheorem{definition}{Definice}

\department{Katedra teoretické informatiky}
\title{Heuristiky pro propagaci intervalů}
\authorGN{Jakub}
\authorFN{Kottnauer}
\authorWithDegrees{Jakub Kottnauer}
\supervisor{doc. Ing. Stefan Ratschan}
\acknowledgements{Doplňte, máte-li komu a za co děkovat. V~opačném případě úplně odstraňte tento příkaz.}
\abstractCS{Cílem práce je otestování vlivu heuristik na~efektivitu řešení numerických CSP pomocí propagací intervalů. Problém splnitelnosti omezujících podmínek (CSP) je problém o~nalezení hodnot proměnných tak, aby byly splněny všechny dané podmínky. Jako součást práce byl napsán řešič v~jazyce F\# implementující algoritmus HC3 s~podporou pro podmínky s~operacemi sčítání, odčítání a násobení. Byly nalezeny podstatné parametry algoritmu ovlivňující jeho účinnost a~následně byly s~\textbf{DOPLNIT POČET} heuristikami provedeny výpočetní experimenty a~jejich výsledky porovnány. Výstup práce bude možno využít při rozvažování, kterou heuristiku použít při řešení soustav omezujících podmínek převeditelných na~podmínky s~výše uvedenými operacemi.}
\abstractEN{Sem doplňte ekvivalent abstraktu Vaší práce v~angličtině.}
\placeForDeclarationOfAuthenticity{V~Praze}
\declarationOfAuthenticityOption{4} %volba Prohlášení (číslo 1-6)
\keywordsCS{propagace intervalů, algoritmus HC3, problém splnitelnosti, omezující podmínky, heuristiky, konzistenční techniky, funkcionální programování, FSharp}
\keywordsEN{Nahraďte seznamem klíčových slov v angličtině oddělených čárkou.}

\begin{document}

% \newacronym{CVUT}{{\v C}VUT}{{\v C}esk{\' e} vysok{\' e} u{\v c}en{\' i} technick{\' e} v Praze}
% \newacronym{FIT}{FIT}{Fakulta informa{\v c}n{\' i}ch technologi{\' i}}

\begin{introduction}
TODO: ro
\end{introduction}

\chapter{Popis problematiky}
\chapter{Cíl práce}
Cílem práce je vytvoření jednoduchého řešiče pro NCSP ve funkcionálním jazyce \texttt{F\#} komunikujícího s uživatel přes příkazovou řádku, nalezení částí algoritmu, které ovlivňují jeho účinnost a rychlost, implementace heuristik nalezených v literatuře či vlastních nápadů (v jakém pořadí je nejlepší domény zužovat, atp.) a experimentálně porovnat účinnost jednotlivých heuristik.

\chapter{Návrh řešení}
\chapter{Realizace}
\chapter{Výpočetní experimenty}

\begin{conclusion}
	%sem napište závěr Vaší práce
\end{conclusion}

\bibliographystyle{csn690}
\bibliography{literatura}

\appendix

\chapter{Seznam použitých zkratek}
% \printglossaries
\begin{description}
	\item[GUI] Graphical user interface
	\item[XML] Extensible markup language
\end{description}

\chapter{Obsah přiloženého CD}

\begin{figure}
	\dirtree{%
		.1 readme.txt\DTcomment{stručný popis obsahu CD}.
		.1 exe\DTcomment{adresář se spustitelnou formou implementace}.
		.1 src.
		.2 impl\DTcomment{zdrojové kódy implementace}.
		.2 thesis\DTcomment{zdrojová forma práce ve formátu \LaTeX{}}.
		.1 text\DTcomment{text práce}.
		.2 thesis.pdf\DTcomment{text práce ve formátu PDF}.
		.2 thesis.ps\DTcomment{text práce ve formátu PS}.
	}
\end{figure}

\end{document}
