\externaldocument{theory.tex}



\chapter{Implementace}
Jako součást bakalářské práce byl vytvořen řešič numerických CSP s~názvem \emph{HullSolver}, který jako jediný volně dostupný řešič umožňuje porovnávat efektivitu použitých heuristik. Zdrojové kódy jsou dostupné pod licencí GNU GPL na~serveru GitHub\footnote{\url{https://github.com/jakubkottnauer/hull-solver}}, případně jsou k nalezení i na~přiloženém CD. Manuál popisující kompilaci, spuštění a použití programu je uveden na~konci práce v~příloze \ref{hullSolverManual}.

Tato část práce nejprve popisuje použité technologie a~algoritmy v~programu HullSolver, následně se věnuje popisu architektury programu a na~závěr uvádí několik existujících řešení, které řeší podobné problémy.


\section{Použité technologie}
Jako jazyk pro implementaci byl zvolen funkcionální jazyk \texttt{F\#} (čteno jako F Sharp) běžící na~platformě .NET, který se v~posledních letech stal velmi populární pro vědecké využití. Tento jazyk byl poprvé uveden v~roce 2005 společností Microsoft, v~roce 2013 byla ale založena nezisková společnost \texttt{F\#} Software Foundation, která má na~starosti jeho další vývoj. Od~té doby se z~jazyka stal open-source a díky projektu Mono je možné aplikace napsané v~\texttt{F\#} spouštět i na~jiných platformách, než pouze na~Microsoft Windows.

\section{Algoritmy}
\label{ch:algorithms}
S~vedoucím práce bylo dohodnuto, že řešič bude podporovat omezující podmínky s~operacemi sčítání, odčítání a násobení.

Řešení vstupních problémů probíhá s~využitím algoritmu HC3 s~malou úpravou pro pestřejší možnosti využití heuristik. Zatímco originální algoritmus vždy vybere z~množiny podmínek jednu náhodnou a zredukuje domény všech proměnných v~ní obsažených, upravený algoritmus na vstupu přijímá množinu všech dvojic $(c, x)$ (kde $c$ je omezující podmínka definovaná nad proměnnou $x$) a z~této množiny heuristicky vybírá jednu dvojici. Následně podle podmínky $c$ zúží doménu proměnné $x$. Omezující podmínka $c$ se tedy na~vstupu objeví tolikrát, nad~kolika proměnnými je nadefinována - v~případě programu HullSolver vždy právě třikrát, neboť program podporuje pouze ternární omezující podmínky. Algoritmus~\ref{HC3AlgorithmAltered} uvádí pseudokód takto upraveného algoritmu HC3.

Oba algoritmy nalézající se v~této kapitole přijímají seznam dvojic omezujících podmínek a proměnných z~CSP, a množinu domén proměnných z~CSP. Tyto parametry spojené dohromady odpovídají CSP problému, jen je u~těchto algoritmů praktičtější definovat CSP v~této alternativní formě.

\begin{algorithm}
\caption{Upravený algoritmus HC3}
\label{HC3AlgorithmAltered}
\begin{algorithmic}[1]
\Require Seznam $P$ dvojic $(c, x)$, kde $c$ je omezující podmínka definovaná nad proměnnou $x$, a množina domén $D$ z CSP.
\Ensure Informace o nekonzistenci, nebo zmenšený problém $P$.
\Procedure{HC3}{$P, D$}
\State $Q \gets P$
\While{$Q \neq \emptyset$ }
\State Pomocí heuristiky vyber z $Q$ pár $p = (c, x)$. \label{HC3AlgorithmAltered:heuristicLine}
\State $Q \gets Q \setminus \left\{ p \right\}$
\State $D_x' \gets$ Zredukuj doménu $D_x \in D$ proměnné $x \in p$ podle $c$. \label{HC3AlgorithmAltered:reduce}
\If{$D_x' = \emptyset $}
\State \Return CSP je nekonzistentní.
\EndIf
\If{$D_x \neq D_x'$} \label{HC3AlgorithmAltered:comparisonLine}
\State $D_x \gets D_x'$
\State Přidej do $Q$ pár $p$ a všechny další dvojice z $P$, které obsahují $x$.
\EndIf
\EndWhile
\State \Return $P$
\EndProcedure
\end{algorithmic}
\end{algorithm}

Na řádku č.~\ref{HC3AlgorithmAltered:comparisonLine} v algoritmu \ref{HC3AlgorithmAltered} jsou pro rovnost porovnávány dvě domény. Vzhledem k tomu, že jsou zde domény tvořeny reálnými intervaly, nemusí být jejich meze celá čísla a v počítači tak nebudou uložena přesně (v programu je pro reprezentaci mezí využit 64-bitový typ \verb|double|). Rovnost je v programu ověřována nepřesně, jak ukazuje následující kód (jako \verb|ZERO_EPSILON| je použita hodnota $10^{-30}$):

\begin{Verbatim}[samepage=true]
member this.EqualTo y =
  abs(this.a - y.a) < ZERO_EPSILON 
  && abs(this.b - y.b) < ZERO_EPSILON
\end{Verbatim}

Nad algoritmem HC3 je postaven jednoduchý branch-and-prune algoritmus (viz Algoritmus \ref{BranchPrune}), který zároveň tvoří hlavní funkci řešiče. Tento algoritmus přibližně kopíruje strukturu algoritmu~\ref{alg:GeneralSolutionAlg} (str.~\pageref{alg:GeneralSolutionAlg}). Při spuštění je mu předána množina dvojic omezujících podmínek a proměnných v~nich obsažených.

\begin{algorithm}
\caption{Upravený algoritmus Solve}
\label{BranchPrune}
\begin{algorithmic}[1]
\Require Seznam $P$ dvojic $(c, x)$, kde $c$ je omezující podmínka definovaná nad proměnnou $x$, a množina domén $D$ z CSP.
\Ensure Seznam nalezených boxů obsahující všechna řešení.
\Procedure{Solve}{$P, D$}
\If{Problém není dostatečně malý} \label{BranchPrune:smallEnough}
\State $P' \gets HC3(P)$
\State $(P_1, P_2) \gets $ rozděl box tvořený doménami dominantních proměnných z $D$ na poloviny. \label{BranchPrune:splitLine}
\State Solve($P_1$)
\State Solve($P_2$)
\Else
\State Vypiš/ulož box tvořený proměnnými z $P$ jako jeden z výsledků.
\EndIf
\EndProcedure
\end{algorithmic}
\end{algorithm}

V použité variantě algoritmu \verb|Solve| je kontrolováno, zda je problém již dostatečně zmenšen (řádek č.~\ref{BranchPrune:smallEnough}). V~programu se konkrétně kontroluje, jestli se již podařilo domény všech dominantních proměnných dostatečně zúžit vzhledem k původní velikosti. Jako koeficient pro porovnání se využívá parametr \verb|eps|, který je specifikován uživatelem při spuštění řešiče přepínačem \verb|-p| (viz manuál).

\begin{Verbatim}[samepage=true]
member this.AllFraction eps =
  dominantVariables 
  |> List.forall(fun v -> 
    (v.Domain.Length / v.OriginalDomain.Length) < eps)
\end{Verbatim}


Druhým prvkem algoritmu \verb|Solve|, jehož implementace je potřeba upřesnit, je rozpůlení boxu na~řádku č.~\ref{BranchPrune:splitLine}. Box, jak je popsáno v kapitole \ref{ch:ncsp}, ohraničuje řešení a je tvořen doménami proměnných z~CSP. Algoritmus box rozpůlí tím, že rozpůlí doménu jedné dominantní proměnné. Ve~vstupním souboru uživatel označí dominantní proměnné a algoritmus je půlí metodou round-robin ve~stejném pořadí, jako jsou zapsány ve~vstupním souboru (při rozpůlení poslední dominantní proměnné přeskočí opět na~první).


\subsection{Efektivita algoritmů}
Nejslabším místem použité varianty algoritmu HC3 je řádek č.~\ref{HC3AlgorithmAltered:heuristicLine}, protože výpočet složité heuristiky může algoritmus velmi zpomalit. Druhým potenciálně rizikovým místem je řádek č.~\ref{HC3AlgorithmAltered:reduce}, protože závisí na konkrétní implementaci zužovací funkce. V~aktuální verzi programu HullSolver by ale nemělo docházet k~problémům díky tomu, že podporuje pouze sčítání a násobení v~omezujících podmínkách, a ty se dají zužovat podle jednoduchých pravidel (navíc je přítomna optimalizace pro zužování podle druhé mocniny).

U branch-and-prune algoritmu \verb|Solve| jsou rovněž dvě potenciálně riziková místa, a to řádky č.~\ref{BranchPrune:smallEnough} a~\ref{BranchPrune:splitLine}. Konkétní implementace popsána výše by ale neměla způsobovat žádné výkonnostní problémy.


\section{Architektura}
Celý program se skládá ze čtyř souborů - \verb|DomainTypes.fs|, \verb|Heuristics.fs|, \verb|Solver.fs|, \verb|Program.fs|. Při kompilování programu napsaného v~\texttt{F\#} záleží na~pořadí souborů a v~tomto případě jsou soubory zpracovávany v~tomto pořadí. Braním ohledu na~pořadí souborů kompilátor zabraňuje tvorbě kruhových závislostí mezi částmi programu, protože soubor později ve frontě může využívat pouze typy z~již zpracovaného souboru. Kód se tak stává modulárnější a přehlednější, protože přirozeně dochází ke striktnímu oddělení \uv{high-level} částí programů od \uv{low-level} částí.

Dalším důsledkem je odlišná organizace kódu v~podobných jazycích jako \texttt{F\#} od klasických procedurálních či objektově orientovaných jazyků. V~\texttt{F\#} se typicky nevytváří samostatný soubor pro každý typ (typ přibližně odpovídá třídě z~objektově orientovaného programování), ale všechny typy se shlukují do jednoho modulu často nazývaného \verb|DomainTypes|. V~případě HullSolveru jsou doménovými typy \verb|Interval|, \verb|Variable|, \verb|Constraint|, \verb|Problem| a~\verb|Heuristic|, z~nichž poslední se pro~přehlednost nachází v~samostatném souboru \verb|Heuristics.fs|.

Zatímco první dva soubory obsahují především deklarace typů a definují jejich funkce, soubor \verb|Solver.fs| tvoří hlavní výpočetní jádro programu - zde jsou implementace branch-and-prune algoritmu a algoritmu HC3. Posledním souborem je \verb|Program.fs|, který se už stará jen o~ošetření a zpracování vstupu a spouští řešící algoritmy.

\section{Použité heuristiky}
\label{ch:usedHeuristics}
Níže jsou uvedeny heuristiky, které jsou využité v~programu HullSolver v algoritmu č.~\ref{HC3AlgorithmAltered} na řádce~\ref{HC3AlgorithmAltered:heuristicLine}. Každá heuristika má svůj krátký název, kterým je identifikována ve~výsledcích experimentů. Heuristiky převzaté z~literatury jsou označeny odkazem na~použitý zdroj.

Všechny heuristiky se chovají stejně v~tom, že v~případě existence více než jedné dvojice splňující jejich kritéria vždy vyberou dvojici nacházející se v~seznamu $Q$ jako první (prohledávání seznamu probíhá zleva doprava).

\begin{itemize}
  \item \emph{rand} je \uv{heuristika} vybírající pseudonáhodnou dvojici,
  \item \emph{fifo} simuluje FIFO frontu - vybírá dvojici, která je ve frontě nejdéle,
  \item \emph{dom/nondom-first} \cite{feiten10} se snaží nejprve redukovat domény dominantních, resp. nedominantních, proměnných - proměnné vyskytující se v~původních nerozložených podmínkách,
  \item \emph{small/large-int-first} \cite{feiten10} se snaží dále zúžit nejužší, resp. nejširší domény,
  \item \emph{shrunk-most/least-first} \cite{feiten10} vybírá nejprve proměnné, jejichž domény byly nejvíce, resp. nejméně, od začátku běhu algoritmu zmenšeny (vypočítává se kvocient z~původní a aktuální velikosti domény),
  \item \emph{max-right-cand} \cite{feiten10} vybírá proměnnou s~nejvyšší pravou mezí domény. Např. proměnná s~doménou $(1;6)$ může být vybrána dříve než proměnná s~doménou $(1;3)$, protože má vyšší pravou mez,
  \item \emph{min-right-cand} \cite{feiten10} vybírá proměnnou s~nejmenší pravou mezí domény,
  \item \emph{fail-first} vybírá proměnnou, která je přítomna v~největším počtu omezujících podmínek,
  \item \emph{prefer-add/mult} vybírá omezující podmínku s~operací sčítání, resp. násobení.
\end{itemize}

\section{Výstup programu a indikátory}
\label{ch:indicators}
V průběhu řešení problémů je sledováno několik tzv.~indikátorů. To jsou různé zajímavé metriky, pomocí kterých se dá následně porovnat efektivita heuristik. Sledovány jsou následující indikátory:
\begin{itemize}
    \item doba běhu,
    \item počet rozpůlení řešení,
    \item počet zúžení intervalů (kolikrát byla zavolána funkce pro zúžení domény podle podmínky),
    \item poměr objemu řešení k objemu vstupního CSP.
\end{itemize}

Doba běhu je závislá na~konkrétním hardware a může se měnit během jednotlivými spuštěními programu, pro hrubé porovnání heuristik ale poslouží dobře.


\section{Možná vylepšení programu}
Program v~této době neumí sám dekomponovat omezující podmínky v~komplexním tvaru a je proto nutné je zadávat v~primitivním tvaru. HullSolver rovněž neprovádí detekci redundantních podmínek.

Dalším vylepšením by mohlo být přidání grafického rozhraní (vykreslování grafů s řešením).






\section{Existující řešení}
Vzhledem k obecnosti termínu CSP existuje velké množství aplikací řešících problémy s~omezujícími podmínkami - lze nalézt nespočet řešičů sudoku i řešičů SAT problému. Vzniklo také několik obecných řešičů, například \emph{Gecode}\footnote{http://www.gecode.org} a \emph{Microsoft Solver Foundation}\footnote{https://msdn.microsoft.com/en-us/library/ff524509(v=vs.93).aspx}.

Řešení NCSP problémů pomocí propagace intervalů je již poměrně specifická oblast a tak těchto řešičů není mnoho. Na~službě GitHub patří mezi nejznámější řešiče projekt \emph{JaCoP}\footnote{\url{https://github.com/radsz/jacop}}, dále existují například \emph{IASolver}\footnote{\url{http://www.cs.brandeis.edu/~tim/Applets/IAsolver.html}}, \emph{RSolver}\footnote{\url{http://rsolver.sourceforge.net}} a \emph{RealPaver}\footnote{\url{https://github.com/lcgutierrez/Realpaver-0\_4-Windows}}. První dva projekty jsou napsány v~Javě, třetí v~jazyce OCaml a čtvrtý v~C. Žádný z~nich ale není dále vyvíjen.
