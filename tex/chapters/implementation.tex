\externaldocument{theory}



\chapter{Implementace}
Součástí bakalářské práce byl vytvořen řešič CSP s názvem \emph{HullSolver}, jehož zdrojové kódy jsou dostupné pod licencí GNU GPL na GitHubu\footnote{\url{https://github.com/jakubkottnauer/hull-solver}}. Manuál popisující kompilaci, spuštění a použití programu je uveden v příloze \ref{hullSolverManual}.

Tato část práce nejprve popisuje použité technologie a algoritmy v programu HullSolver, následně se věnuje popisuje architektury programu a na závěr uvádí několik existujících řešení, které řeší podobné problémy.


\section{Použité technologie}
Jako jazyk pro implementaci byl zvolen funkcionální jazyk F\# (čteno F Sharp) běžící na platformě .NET, který se v posledních letech stal velmi populární pro vědecké využití. Tento jazyk byl poprvé uveden v roce 2005 společností Microsoft, v roce 2013 byla ale založena nezisková společnost F\# Software Foundation, která má na starosti jeho další vývoj. Od té doby se z jazyka stal open-source a díky projektu Mono je možné aplikaci napsané v F\# spouštět i na jiných platformách, než pouze na Microsoft Windows.


\subsection{Funkcionální programování v F\#}
(trochu popsat immutability, types, modules v F\#...)

\section{Algoritmy}

S vedoucím práce bylo dohodnuto, že řešič bude podporovat omezující podmínky s operacemi sčítání, odčítání a násobení.

Řešení vstupních problémů probíhá s využitím algoritmu HC3 s mírnou úpravou pro pestřejší možnosti využití heuristik. Zatímco originální algoritmus vždy vybere z množiny podmínek jednu náhodnou a zredukuje domény všech proměnných v ní obsažených, upravený algoritmus heuristicky vybere dvojici $(c, x)$, kde $c$ je omezující podmínka a $x$ je proměnná obsažená v $c$, a podle této podmínky zúží doménu proměnné $x$. Algoritmus~\ref{HC3AlgorithmAltered} uvádí pseudokód takto upraveného algoritmu HC3.

\begin{algorithm}
\caption{Upravený algoritmus HC3}
\label{HC3AlgorithmAltered}
\begin{algorithmic}[1]
\Require Seznam dvojic $(c, x)$, kde $c$ je omezující podmínka definovaná nad proměnnou $x$.
\Ensure Informace o nekonzistenci, nebo box obalující řešení.
\Procedure{HC3}{$P$}
\State $Q \gets P$
\While{$Q \neq \emptyset$ }
\State Pomocí heuristiky vyber z $Q$ pár $p = (c, x)$.
\State $Q \gets Q \setminus \left\{ p \right\}$
\State $D_x' \gets$ Zredukuj doménu $D_x$ proměnné $x$ podle podmínky $c$.
\If{$D_x' = \emptyset $}
\State Konec - CSP je nekonzistentní.
\EndIf
\If{$D_x \neq D_x'$}
\State $D_x \gets D_x'$
\State Přidej zpět do $Q$ pár $p$ a všechny další páry z $P$, které obsahují proměnnou $x$.
\EndIf
\EndWhile
\EndProcedure
\end{algorithmic}
\end{algorithm}

Nad algoritmem HC3 je postaven jednoduchý branch and prune algoritmus (viz Algoritmus \ref{BranchPrune}), který zároveň tvoří hlavní funkci řešiče. Tento algoritmus přibližně kopíruje strukturu algoritmu~\ref{alg:GeneralSolutionAlg} (str.~~\pageref{alg:GeneralSolutionAlg}), jen je zde využit rekurzivní zápis. Při spuštění je mu předána množina dvojic omezujících podmínek a proměnných v nich obsažených.

\begin{algorithm}
\caption{Branch \& Prune}
\label{BranchPrune}
\begin{algorithmic}[1]
\Require Seznam dvojic $(c, x)$, kde $c$ je omezující podmínka definovaná nad proměnnou $x$.
\Ensure Seznam nalezených boxů.
\Procedure{Solve}{$P$}
\If{Problém není dostatečně malý}
\State $P' \gets HC3(P)$
\State $(P_1, P_2) \gets $ rozděl box tvořený proměnnými z $P'$ na poloviny.
\State Solve($P_1$)
\State Solve($P_2$)
\Else
\State Vypiš/ulož box tvořený proměnnými z $P$ jako jeden z výsledků.
\EndIf
\EndProcedure
\end{algorithmic}
\end{algorithm}

\section{Architektura}
Celý program se skládá ze čtyř souborů - \verb|DomainTypes.fs|, \verb|Heuristics.fs|, \verb|Solver.fs|, \verb|Program.fs|. Při kompilování programu napsaného v F\# záleží na pořadí souborů a v tomto případě jsou soubory zpracovávany v tomto pořadí. Braním ohledu na pořadí souborů kompilátor zabraňuje tvorbě kruhových závislostí mezi částmi programu, protože soubor později ve frontě může využívat pouze typy z již zpracovaného souboru. Kód se tak stává modulárnější a přehlednější, protože přirozeně dochází ke striktnímu oddělení \uv{high-level} částí programů od \uv{low-level} částí.

Dalším důsledkem je odlišná organizace kódu v podobných jazycích jako F\# od klasických procedurálních či objektově orientovaných jazyků. V F\# se typicky nevytváří samostatný soubor pro každý typ, ale všechny typy se shlukují do jednoho modulu často nazývaného \verb|DomainTypes|. V případě HullSolveru jsou doménovými typy \verb|Interval|, \verb|Variable|, \verb|Constraint|, \verb|Problem| a \verb|Heuristic|, z nichž poslední se pro přehlednost nachází v samostatném souboru \verb|Heuristics.fs|.

Zatímco první dva soubory obsahují především deklarace typů a definují jejich funkce, soubor \verb|Solver.fs| tvoří hlavní výpočetní jádro programu - zde jsou implementace Branch \& Prune algoritmu a algoritmu HC3. Posledním souborem je \verb|Program.fs|, který se už stará jen o ošetření a zpracování vstupu a spouští řešící algoritmy.

\section{Výstup programu a indikátory}
V průběhu řešení problémů je sledováno několik tzv. indikátorů. To jsou různé zajímavé metriky, pomocí kterých se dá následně porovnat efektivita heuristik. Kromě času běhu algoritmu (který je ale spíše pouze ilustrativní, protože se může měnit mezi jednotlivými spuštěními) jsou sledovány následující indikátory:

\begin{itemize}
    \item Počet rozpůlení řešení
    \item Počet zúžení intervalů
    \item Poměr objemu řešení k objemu vstupního CSP
\end{itemize}


\section{Existující řešení}
Vzhledem k obecnosti termínu CSP existuje velké množství aplikací řešících problémy s omezujícími podmínkami - najdeme nespočet řešičů sudoku i řešičů SAT problému. Vzniklo také několik obecných řešičů, například \emph{Geocode}\footnote{http://www.gecode.org} a \emph{Microsoft Solver Foundation}\footnote{https://msdn.microsoft.com/en-us/library/ff524509(v=vs.93).aspx}.

Řešení NCSP problémů pomocí propagace intervalů je již poměrně specifická oblast a tak těchto řešičů není mnoho. Na službě GitHub patří mezi nejznámější řešiče projekt \emph{JaCoP}\footnote{\url{https://github.com/radsz/jacop}}, dále existují například \emph{IASolver}\footnote{\url{http://www.cs.brandeis.edu/~tim/Applets/IAsolver.html}}, \emph{RSolver}\footnote{\url{http://rsolver.sourceforge.net}} a \emph{RealPaver}\footnote{\url{https://github.com/lcgutierrez/Realpaver-0\_4-Windows}}. První dva projekty jsou napsány v Javě, třetí v jazyce OCaml a čtvrtý v C. Žádný z nich ale není dále vyvíjen.
