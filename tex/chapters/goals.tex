\chapter{Cíl práce}
V první části byla vysvětlena veškerá potřebná terminologie a je tedy nyní možné přesněji popsat cíle práce. Hlavním cílem práce je vytvoření jednoduchého řešiče pro~numerické CSP problémy ve~funkcionálním jazyce \texttt{F\#}. Program komunikuje s~uživatelem přes příkazovou řádku a~problémy řeší s~využitím branch-and-prune algoritmu a~algoritmu HC3.

Dalším cílem práce je nalezení parametrů algoritmu HC3, které ovlivňují jeho účinnost a~rychlost. V~řešiči jsou implementovány heuristiky (nalezené v~literatuře či vymyšlené), které mají za účel pomoci rozhodnout, v~jakém pořadí je nejlepší domény zužovat, atp. Posledním cílem práce je experimentálně porovnat účinnost jednotlivých heuristik na~sadě benchmarkových úloh a zjištěné výsledky vyhodnotit.