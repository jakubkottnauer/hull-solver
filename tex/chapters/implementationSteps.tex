\chapter{Návrh řešení}
\section{Existující řešení}
Vzhledem k obecnosti termínu CSP existuje velké množství aplikací řešících problémy s omezujícími podmínkami - najdeme nespočet řešičů sudoku i řešičů známého SAT problému\footnote{Boolean satisfiability problem}. Vzniklo také několik obecných řešičů, například \emph{Geocode}\footnote{http://www.gecode.org} a \emph{Microsoft Solver Foundation}\footnote{https://msdn.microsoft.com/en-us/library/ff524509(v=vs.93).aspx}.

Řešení NCSP problémů pomocí propagace intervalů je již poměrně specifická oblast a tak těchto řešičů není mnoho. Na službě GitHub patří mezi nejznámější řešiče projekt \emph{JaCoP}\footnote{https://github.com/radsz/jacop}, dále existují například \emph{IASolver}\footnote{http://www.cs.brandeis.edu/~tim/Applets/IAsolver.html}, \emph{RSolver}\footnote{http://rsolver.sourceforge.net/} a \emph{RealPaver}\footnote{https://github.com/lcgutierrez/Realpaver-0\_4-Windows}. První dva projekty jsou napsány v Javě, třetí v jazyce OCaml a čtvrtý v C. Žádný z nich ale není dále vyvíjen.
