\begin{algorithm}
\caption{Branch \& Prune}
\label{BranchPrune}
\begin{algorithmic}[1]
\Require Množina dvojic $(c, x)$, kde $c$ je omezující podmínka definovaná nad proměnnou $x$.
\Ensure Seznam nalezených boxů.
\Procedure{Solve}{$P$}
\If{Problém není dostatečně malý}
\State $P' \gets HC3(P)$
\State $(P_1, P_2) \gets $ rozděl box tvořený proměnnými z $P'$ na poloviny.
\State Solve($P_1$)
\State Solve($P_2$)
\Else
\State Vypiš/ulož box tvořený proměnnými z $P$ jako jeden z výsledků.
\EndIf
\EndProcedure
\end{algorithmic}
\end{algorithm}

V algoritmu \ref{BranchPrune} jsou dva řádky, které je nutné dále rozebrat - podmínku pro ukončení algoritmu (řádek č.~2) a~způsob rozdělení problému na~poloviny (řádek č.~4).

Pro ukončení algoritmu je potřeba stanovit mezní velikost, pod kterou se musí domény dominantních proměnných zúžit. Tato mez může být absolutní (velikost domény maximálně například 0,01), nebo relativní vzhledem k původním nezúženým doménám (například požadavek za zmenšení domény na tisícinu původní velikosti).


